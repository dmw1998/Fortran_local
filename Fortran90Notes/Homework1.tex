\documentclass[a4paper,titlepage]{report}
\usepackage{amsmath}
\usepackage{amssymb}
\usepackage{graphicx}
\usepackage{caption}
\usepackage{float}
\usepackage{listings}
\usepackage{color}
\usepackage{xeCJK}

\title{Fortran 90 Reading Notes}
\author{Dong Miaowen}
\date{2020.06.23-}

\begin{document}
\maketitle

\tableofcontents{}

\chapter{第一章略}

\section{章节标题}

\subsection{小标题}


\chapter{First steps in Fortran 90 programming}

\section{From problem to program in three basic steps}

(1) 明确问题 \\
(2) 分析问题并将其分解为几个基本元素 \\
(3) 根据上步结果编写代码 \\
(4) 测试——重复二、三步直至得到好的结果 \\

\section{Some basic Fortran 90 concepts}

\subsection*{Self-test Exercises 2.1}

\begin{itemize}
  \item[1.]
  (1) 明确问题 \\
  (2) 分析问题并将其分解为几个基本元素 \\
  (3) 根据上步结果编写代码

  \item[2.]
  测试

  \item[3.]
  以字母开头、不包含特殊符号、长度小于31个字符、不区分大小写。

  \item[4.]
  program name \\
  end program name

  \item[5.]
  implicit none \\
  用于禁止一些在更老版本的Fortran中出现的一些内容。 \\ \\

  Answer: \\
  If the last non-blank character of a line is an ampersand (\&), then the statement is continued on the next time. \\
  If the ampersand appears in a character context(that is, in the middle of a character string enclosed in quotation marks or apostrophes), then the first non-blank character of the next line must also be an ampersand, and the character string continues from immediately after that ampersand. \\
  If the ampersand at then end of the first line is not in a character context, then the statment is continued either from the first character after an ampersand, if that is the first non-blank character on the line, or from the start of the next line, if the first non-blank character is not an ampersand.

  \item[6.]
  \! 后跟 comments 用于提示、解释代码作用。 \\
  在整段代码前或语句结束后的任意位置可以添加。

\end{itemize}

\section{Running Fortran programs on a computer}

\section{Errors in programs}

\section{The design and testing of programs}

编写程序时,我们需要考虑:
\begin{itemize}
  \item[1.]
  Elegance. 花更多的时间在设计程序上,以更好地编写更优美的代码。强调易读性和效率。

  \item[2.]
  Maintianability. 比起编写新的算法,编写长期可维护的算法更加重要。Remembering "write once and read many times".

  \item[3.]
  Portable programs. 程序最终可以适应于大部分计算机。

\end{itemize}

好的程序需要:
\begin{itemize}
  \item[1.]
  完全掌握程序的目的,确认好所有 input 和 output 。

  \item[2.]
  使 input 尽可能简单易理解; 使 output 清晰明确有用。

  \item[3.]
  清楚地写下解决问题的方案,将整个问题划分为更加好实现的子问题。以便后期修改直至得到正确答案。

  \item[4.]
  查看已有的可用代码 in procedure libraries 。

  \item[5.]
  Using a modular design to make sure that no single block of code should be longer than about 50 lines, excluding any comments.

  \item[6.]
  Use descriptive namees for varibles and program units anf be lavish with comments.

  \item[7.]
  Input error checking.

  \item[8.]
  测试程序的每一部分。

\end{itemize}

\section{The old and new Fortran 90 source forms}

\subsection*{Self-text Exercises 2.2}

\begin{itemize}
  \item[1.]
  Syntactic error: 语法错误,包括拼写错误和不符合语法规范等。
  Semantic error: 语法没有错误,但不符合逻辑。

  Syntactic error 可被 compiler 检测出,导致 compilation errors 。Semantic errors 通常不会被检测出来,属于逻辑硬伤,导致 execution errors 。

  \item[2.]
  Elegance(easier to test), maintianability, portability.

  \item[3.]
  \begin{itemize}
    \item[1]
    完全掌握程序的目的,确认好所有 input 和 output 。

    \item[2]
    使 input 尽可能简单易理解; 使 output 清晰明确有用。

    \item[3]
    清楚地写下解决问题的方案,将整个问题划分为更加好实现的子问题。以便后期修改直至得到正确答案。

    \item[4]
    查看已有的可用代码 in procedure libraries 。

  \end{itemize}

  \item[4.]
  Answer: \\
  \begin{itemize}
    \item[(1)]
    Ensure that your program carries out as many checks on the validity of the data it reads as is possible (and realistic). A program that attempts to produce a meaningful answer.

    \item[(2)]
    Carry out internal valisity checks at critical points in the calculations.

    \item[(3)]
    Check that a reasonable number of iterations aree being made while trying to converege to a solution.

    \item[(4)]
    Test each part of your program thoroughly before testing the complete grogram.

  \end{itemize}

  \item[5.]
  132

  \item[6.]
  40

  \item[7.]
  No limit to the number of statements, but of characters. \\
  By at least one blank (semi-colons).

  \end{itemize}

\end{document}
